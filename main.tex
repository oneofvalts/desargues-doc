\documentclass[10pt, oneside]{memoir}

\newlength{\pagew}
\newlength{\pageh}
\setlength{\pagew}{210mm}
\setlength{\pageh}{297mm}

\setlength{\headheight}{30pt}

\setstocksize{\pageh}{\pagew}
\settrimmedsize{\pageh}{\pagew}{*}
\settypeblocksize{170mm}{110mm}{*}
\setlrmargins{*}{*}{1}
\setulmargins{*}{*}{1}
\setmarginnotes{0cm}{0cm}{0cm}
\checkandfixthelayout

% \usepackage[pass, showframe]{geometry}

\usepackage[french]{babel}

% \usepackage{tgpagella}
% \usepackage[T1]{fontenc}

\usepackage{mathtools}
\usepackage[mathscr]{eucal}
\usepackage{amsmath}
\usepackage{amsthm}
\usepackage{physics}
% \usepackage{newpxmath}

\usepackage[math-style=ISO, bold-style=ISO]{unicode-math}
\setmainfont{EB Garamond}
\setmathfont{Garamond-Math.otf}[StylisticSet={7,9}]
\setmonofont[Scale=MatchLowercase]{Iosvmata}
\usepackage{microtype}

\usepackage[nomarkers, figuresonly]{endfloat}

% \usepackage[no-math]{fontspec}
% \setmainfont{EB Garamond}
% \setmonofont[Scale=MatchLowercase]{JetBrains Mono}
%
% \usepackage{microtype}
%
% \usepackage{mathspec}
% \setmathrm{EB Garamond}
% \setmathfont(Digits,Greek,Latin){EB Garamond}
%
% \makeatletter % undo the wrong changes made by mathspec
% \let\RequirePackage\original@RequirePackage
% \let\usepackage\RequirePackage
% \makeatother

\usepackage{graphicx}
\usepackage{svg}
\usepackage{minted}

\newmintinline[lean]{lean}{bgcolor=white}
\newminted[leancode]{lean}{}
\usemintedstyle{tango}

\usepackage{float}
\usepackage{enumitem}
\usepackage{tabto}

\makeatletter
\newcommand\iraggedright{%
  \let\\\@centercr\@rightskip\@flushglue \rightskip\@rightskip
  \leftskip\z@skip}
\makeatother

\usepackage{lipsum}
\usepackage{dirtytalk}

\usepackage{pst-solides3d}
\psset
{
  lightsrc=viewpoint,
  Decran=30,
  solidmemory,
}

\usepackage{tikz}
\usetikzlibrary{intersections}

\usetikzlibrary{external}
\tikzexternalize[prefix=figures/]

% \usepackage{comment}
% \excludecomment{figure}
% \let\endfigure\relax

\usepackage[
backend=biber,
style=alphabetic
]{biblatex}
\addbibresource{main.bib}

\usepackage{csquotes}

\usepackage[
unicode,
colorlinks
]{hyperref}

\usepackage{cleveref}

\counterwithout{figure}{chapter}
\maxtocdepth{subsection}
\setsecnumdepth{section}
\renewcommand\thesection{\arabic{section}}

\newtheorem{thm}{Théorème}
\newtheorem{prop}{Proposition}
\theoremstyle{definition}
\newtheorem{defn}{Définition}
\theoremstyle{remark}
\newtheorem{exm}{Exemple}

% Unbutton the vector space from its origin.
\newcommand{\unbutton}[1][.7]{\mathbin{\hspace{1pt}\vcenter{\hbox{\scalebox{#1}{$\bullet$}}}}}

% Definitional equal.
\newcommand{\defeq}{\vcentcolon=}

% Explicit set and set comprehension.
\newcommand{\set}[1]{\{ #1 \}}
\newcommand{\setcomp}[2]{\set{ #1 \,|\, #2 }}

% Projectivization.
\newcommand{\proj}{\symbfscr{P}}

% Disjoint union.
\newcommand{\discup}{\mathop{\dot{\cup}}}

% Partial map.
\newcommand*{\DashedArrow}[1][]{\mathbin{\tikz
    [baseline=-0.25ex,-latex, dashed,#1] \draw [#1] (0pt,0.5ex) --
    (1.3em,0.5ex);}}%
\newcommand{\partialto}{\DashedArrow[->,dash pattern=on 4pt off 2pt]}

% Kernel and domain.
\DeclareMathOperator{\kernel}{Ker}
\DeclareMathOperator{\domain}{Dom}

% Common sets
\newcommand{\R}{\mathbf{R}}

\begin{document}
\thispagestyle{empty}
\begin{center}
  \vspace*{\fill}
  % {%
  % \setlength{\fboxsep}{0pt}%
  % \fbox{\includesvg[width=3cm]{logo.svg}}%
  % }
  \includesvg[width=2.5cm]{logo.svg}

  \vspace{1cm}
  \large{\textbf{PROJET DE FIN D'ÉTUDES}}

  \vspace{0.5cm}
  {\small pour obtenir le diplôme de}

  \vspace{0.5cm}
  l'\textsc{\textbf{Université Galatasaray}}

  {\small Spécialité : \textbf{Mathématiques}}

  \vspace{2.25cm}
  {\Large\textbf{Formalisation du théorème de Desargues en Lean}}

  Rapport 3

  \vspace{1.25cm}
  Préparé par \textbf{Abdullah Uyu}

  Résponsable : \textbf{Can Ozan Oğuz}

  \vspace{2.25cm}
  \textit{5 novembre 2023}
  \vspace*{\fill}
\end{center}
\clearpage
\tableofcontents
\iraggedright
\pagenumbering{arabic}
\chapter{Introduction}
\paragraph{Motivation}
Lean est un assistant de preuve. La transcription d'une preuve en
langage humain dans un assistant de preuve est appelée
\textit{formalisation}. Lean dispose d'une grande bibliothèque de
preuves, y compris de nombreuses preuves du niveau de licence, appelée
mathlib4. Dans cette bibliothèque, il y a d'importants théorèmes
manquants, et le théorème de Desargues est l'un d'entre eux.
\paragraph{Objectifs du Projet}
Les deux aspects principaux de ce projet sont l'apprentissage des
rudiments de la théorie des géométries projectives, et de la
programmation en Lean. Avec ces deux aspects, le but ultime est de
formaliser le théorème de Desargues dans Lean.
\chapter{Explications sur les résultats réquis de la géométrie projective}
\section{Première rencontre}
Pour motiver les géométries projectives, on commence par considérer
les droites passant par l'origine dans le plan. On peut représenter la
plupart de ces lignes par des points sur l'axe $y=1$.
\begin{figure}[H]
  \centering
  \begin{tikzpicture}
    \draw[-stealth,ultra thick] (-5,0)--(5,0) node[right] {$x$};
    \draw[-stealth,ultra thick] (0,-2)--(0,4) node[above] {$y$};
    \draw[name path=y1,semithick] (-4,1)--(4,1) node[right] {$y=1$};
    \draw[name path=line 1,semithick] (1,-1.5)--(-2,3) {};
    \draw[name path=line 2, semithick] (-2,-1.5)--(4,3) {};
    \fill[name intersections={of=y1 and line 1}] (intersection-1) circle (2pt)
    node[below left] {$(-2/3,1)$};
    \fill[name intersections={of=y1 and line 2}] (intersection-1) circle (2pt)
    node[below right] {$(4/3, 1)$};
  \end{tikzpicture}
  \caption{La représentation des droites passant par l'origine dans le
    plan}
  \label{fig:lines-plane}
\end{figure}
La seule droite que l'on n'a pas réussi à représenter est l'axe
$x$. On notera également que l'on peut bien sûr choisir n'importe quel
axe pour la représentation, à l'exception de ceux qui passent par
l'origine. Cette impossibilité dans les cas d'exception est clairement
visible sur la \autoref{fig:lines-plane}. Si l'on choisit un tel axe,
la droite que l'on ne parviendra pas à représenter sera la droite
(passant par l'origine) qui est parallèle à cet axe.

Effectuons la même procédure pour l'espace. On peut représenter la
plupart des droites passant par l'origine par des points sur le plan
$z=1$.
\begin{figure}[H]
  \centering
  \begin{pspicture}[viewpoint=30 40 30 rtp2xyz] (-5,-2) (5,4)
    \psSolid[object=plan,
    definition=equation,
    args={[0 0 1 0]},
    base=-2 3 -2 2.5]
    \psSolid[object=plan,
    definition=equation,
    args={[0 0 1 -2]},
    base=-2 3 -2 2.5]
    \psSolid[object=point,
    args=0 0 2]
    \psSolid[object=line,
    linestyle=dashed,
    args=0 0 0 -1 0.67 2]
    \psSolid[object=line,
    args=-1 0.67 2 -1.5 1 3]
    \psSolid[object=point,
    args=-1 0.67 2]
    \psSolid[object=line,
    linestyle=dashed,
    args=0 0 0 0.67 -1 2]
    \psSolid[object=line,
    args=0.67 -1 2 1 -1.5 3]
    \psSolid[object=point,
    args=0.67 -1 2]
    \axesIIID[labelsep=10pt] (0.5,0.5,2) (3.5,3.5,3.5)
    \rput(-2,3.05){$(-1/2,1/3,1)$}
    \rput(2.5,3.2){$(1/3,-1/2,1)$}
    \rput(3,1){$z=1$}
  \end{pspicture}
  \caption{La représentation des droites passant par l'origine dans l'espace}
  \label{fig:lines-space}
\end{figure}
Maintenant, les seules droites que nous ne pouvons pas représenter
sont exactement les droites du plan que nous avons représenté
précédemment. La remarque sur le choix de l'axe de représentation
s'étend ici comme tout plan ne passant pas par l'origine peut être
utilisé. De plus, le plan irreprésentable sera celui (qui passe par
l'origine) qui est parallèle au plan de représentation.

Ce processus s'appelle la \textit{projectivisation d'un espace vectoriel}.
Écrivons-le en langage d'algèbre linéaire.
\section{Projectivisation d'un espace vectoriel}
\begin{defn}[{\cite[27]{ff00}}]
  \label{projectivization}
  Soit $V$ un espace vectoriel. Sur
  $V^{\unbutton} \defeq V \setminus \set{0}$, on définit une relation
  binaire comme suit : $x \sim y$ si et seulement si $x, y$ sont
  linéairement dépendants. Comme ceci est une relation
  d'équivalence, l'ensemble quotient
  $\proj(V) \defeq V^{\unbutton} / {\sim}$ est bien défini. $\proj(V)$
  est appelée la \textit{projectivisation de l'espace vectoriel $V$}.
\end{defn}
% Pour simplifier le langage, nous aurons également besoin de la
% définition de l'union disjointe.
% \begin{defn}
%   Soit $A$ et $B$ deux ensembles. L'union
%   $(A \times \set{1}) \cup (B \times \set{0})$, noté $A \discup B$, est
%   appelée \textit{union disjointe} de $A$ et $B$.
% \end{defn}
La motivation ci-dessus peut donc être formulée comme suit :
\begin{align*}
  \label{eq:embedding}
  \proj(\R^3) &= \proj(\R^2 \times \R) \\
              &\cong \R^2 \discup \proj(\R^2) \\
              &= \R^2 \discup \proj(\R \times \R) \\
              &\cong \R^2 \discup \R \discup \proj(\R).
\end{align*}
Cela résume ce que l'on fait mathématiquement lorsque l'on fait des
dessins en perspective : On prend un plan ($\R^2$), on choisit un
horizon ($\R$) et un point de fuite ($\proj(\R)$). La proposition
suivante n'est qu'une généralisation de ce processus.
\begin{prop}[{\cite[28]{ff00}}]
  Pour un $K$-espace vectoriel $V$, il existe une bijection naturelle
  \begin{equation*}
    \label{natural-bijection}
    s: \proj(V \times K) \to V \discup \proj(V)
  \end{equation*}
  induite par la fonction
  $t: (V \times K)^{\unbutton} \to V \discup \proj(V)$ définie par
  $t(x,\xi) = \xi^{-1}x$ si $\xi \neq 0$ et $t(x,0) = [x]$ pour
  $x \neq 0$, où $[x]$ désigne le point de $\proj(V)$ représenté par
  $x$.
  % Il existe donc une relation ternaire unique $\overline{\ell}$ sur
  % $V \discup \proj(V)$ pour laquelle $V \discup \proj(V)$ devient
  % une géométrie projective et $s$ un isomorphisme. De plus, on a:
  % \NumTabs{2}
  % \begin{enumerate}[align=left]
  %   \item $\overline{\ell}(x, y, z)$ \tabto{3cm} ssi $x-z$ et $y-z$
  %   sont linéairement dépendants dans $V$,
  %   \item $\overline{\ell}(x, y, [z])$ \tabto{3cm} ssi $x-y = \mu z$
  %   pour un $\mu \in K$
  %   \item $\overline{\ell}(x, [y], [z])$ \tabto{3cm} ssi $[y] = [z]$,
  %   \item $\overline{\ell}([x], [y], [z])$ \tabto{3cm} ssi
  %   $\ell([x], [y], [z])$.
  % \end{enumerate}
\end{prop}

\begin{defn}[{\cite[26]{ff00}}]
  \label{projective-geometry}
  Une \textit{géométrie projective} est un ensemble $G$ accompagné
  d'une relation ternaire $\ell \subseteq G \times G \times G$ telle
  que les axiomes suivants sont satisfaits :
  \begin{itemize}[align=left]
    \item[($\text{L}_1$)] $\ell(a,b,a)$ pour tout $a, b \in G$.
    \item[($\text{L}_2$)] $\ell(a,p,q)$, $\ell(b,p,q)$ et $p \neq q$
          $\implies$ $\ell(a,b,p)$.
    \item[($\text{L}_3$)] $\ell(p,a,b)$ et $\ell(p,c,d)$ $\implies$
          $\ell(q,a,c)$ et $\ell(q,b,d)$ pour un $q \in G$.
  \end{itemize}
  Les éléments de $G$ sont appelés les \textit{points} de la
  géométrie. Et trois points $a, b, c$ sont dits \textit{colinéaires}
  si $\ell(a,b,c)$.
\end{defn}
% Un système équivalent d'axiomes utilisant l'ensemble, par exemple,
% $a \star b$, créé par $\ell$ est omis ici, par souci de
% concision. La définition de morphisme sera donnée à l'aide de ce
% système.

Notons l'exemple le plus important pour une géométrie projective. Lors d'une
\href{https://leanprover.zulipchat.com/#narrow/stream/113489-new-members/topic/Missing.20theorems.20list/near/397885401}{discussion}
sur la chaîne de Zulip de Lean, Joseph Myers a conseillé d'énoncer le théorème
sur cet exemple, mais pas sur la définition abstraite d'une géométrie
projective. De plus, il est noté que la version du théorème en géométrie
euclidienne, qui est probablement même connue de nombreux élèves du lycée, peut
être déduite de l'énoncé du théorème sur cet important modèle de géométrie
projective.
\begin{prop}[{\cite[27]{ff00}}]
  \label{example}
  Pour un espace vectoriel $V$, $\proj(V)$ est une géométrie
  projective si pour tout élément $X, Y, Z \in \proj(V)$ on définit :
  $\ell(X,Y,Z)$ si et seulement si $X, Y, Z$ ont des représentants
  $x, y, z$ linéairement dépendants.
\end{prop}
\section{Isomorphismes}
Les isomorphismes seront très utiles tout au long du voyage.
\begin{defn}[{\cite[27]{ff00}}]
  Un \textit{isomorphisme} de géométries projectives est une bijection
  $g: G_1 \to G_2$ satisfaisant $\ell_1(a,b,c)$ si et seulement si
  $\ell_2(ga,gb,gc)$. Si $G_1 = G_2$, alors on dit que $g$ est une
  \textit{collinéation}.
\end{defn}
Toutes les applications linéaires bijectives induisent une
colinéation.
\begin{exm}
  Soit $T: \R^n \to \R^n$ une application linéaire
  bijective. L'application $g: \proj(\R^n) \to \proj(\R^n)$,
  $[x] \mapsto [T(x)]$ est une isomorphisme de géometries projectives.

  L'application $g$ est bien définie: Soit $x, y \in \R^n$ tel que
  $[x] = [y]$, i.e. $x = ky$ pour un $k \in \R$. On a:
  \begin{align*}
    [T(x)] &= [T(ky)] \quad x \ \text{définition} \\
           &= [kT(y)] \quad T \ \text{linéaire} \\
           &= [T(y)] \quad \text{\cref{projectivization}.}
  \end{align*}
  D'où, $g$ est bien définie.

  Montrons que $g$ est injective. Soit $[x], [y] \in \proj(\R^n)$ tel
  que $[T(x)] = [T(y)]$, i.e. $T(x) = kT(y)$ pour un $k \in \R$. Comme
  $T$ est linéaire, $T(x) = T(ky)$. De plus, $x = ky$ car $T$ est
  injective. Par la définition de la classe d'équivalence, on obtient
  $[x] = [y]$. D'où $g$ est injective. Pour la surjectivité, prenons
  $[x] \in \proj(\R^n)$. Comme $T$ est bijective, $T^{-1}$ existe, et
  $[T^{-1}(x)] \in \proj(\R^n)$. Ainsi
  $g([T^{-1}(x)]) = [T(T^{-1}(x))] = [x]$. D'où, $g$ est
  surjective. On a montré que $g$ est bijective.

  Vérifions la condition d'isomorphisme. Soit
  $[x], [y], [z] \in \proj(\R^n)$. Supposons que
  $\ell([x], [y], [z])$. On a des équivalences:
  \begin{align*}
    &\iff ax + by + cz = 0 \quad \text{\cref{example}} \\
    &\iff T(ax + by + cz) = 0 \quad T \ \text{linéaire, injective} \\
    &\iff aT(x) + bT(y) + cT(z) = 0 \quad T \ \text{linéaire} \\
    &\iff \ell([T(x)], [T(y)], [T(z)]) \quad \text{\cref{example}}
  \end{align*}
\end{exm}
Mais les colinéations sont bien plus que des applications linéaires bijectives.
En d'autres termes, il existe des collinéations qui ne sont pas induites par une
application linéaire.
\begin{exm}
  L'application $g: \proj(\R^2) \to \proj(\R^2)$ définie par
  $[(x_1, x_2)] \mapsto [((x_{1} / x_{2})^3, 1)]$ si $x_{2}$ est non-nul, et
  $[(x_{1}, 0)] \mapsto [(x_{1}, 0)]$ sinon, est une collineation qui n'est pas
  induite par une application linéaire.

  L'application $g$ est bien définie. Soit $x_{1}, x_{2}, x_{1}', x_{2}'$ des
  rééls; $x_{2}, x_{2'}$ non-nuls, tel que
  $[(x_{1}, x_{2})] = [(x_{1}', x_{2}')]$. Alors, $x_{1}' = kx_{1}$ et
  $x_{2}' = kx_{2}$ pour un $k$ non-nul. On a:
  \begin{align*}
    [((x_{1}' / x_{2}')^{3}, 1)] &= [((kx_{1} / kx_{2})^{3}, 1)] \\
                                 &= [((x_{1} / x_{2})^{3}, 1)]
  \end{align*}
  D'où, $g$ est bien définie.

  Montrons que $g$ est injective. Soit $x_{1}, x_{2}, x_{1}', x_{2}'$ des rééls
  tel que
  $$
  [((x_{1} / x_{2})^{3}, 1)] = [((x_{1}' / x_{2}')^{3}), 1].
  $$
  On va montrer que $[(x_{1}, x_{2})] = [(x_{1}', x_{2}')]$. Par la définition
  de la classe d'équivalence, on a
  $$
  ((x_{1} / x_{2})^{3}, 1) = k ((x_{1} / x_{2}')^{3}, 1)
  $$
  pour un $k$ non-nul. Alors, $k = 1$ et on a:
  $$
  x_{1} / x_{2} = x_{1}' / x_{2}'.
  $$
  Si $x_{1}$ est nul, alors $x_{1}'$ l'est aussi, et donc on a:
  \begin{align*}
    [(0, x_{2})] &= [0, \frac{x_{2}'}{x_{2}} x_{2}] \quad \text{ classe
                   d'équivalence définition} \\
                 &= [0, x_{2}]
  \end{align*}
  Sinon on a $x_{1} / x_{1}' = x_{2} / x_{2}'$ et donc:
  \begin{align*}
    [(x_{1}, x_{2})] &= [(\frac{x_{1}' x_{2}}{x_{2}'}, \frac{x_{1} x_{2}'}{x_{1}'})] \\
                     &= [(x_{1}', x_{2}')] \quad \text{classe d'équivalence définition}
  \end{align*}
  Enfin, on vérifie la reste de l'image. Soit $x_{3}$ un réél. Il est clair que
  $[((x_{1} / x_{2})^{3}, 1)] \neq [(x_{3}, 0)]$. D'où, $g$ est injective. Pour
  la surjectivité, prenons des rééls $x_{1}, x_{2}$. Si $x_{2}$ est nul, on a
  $g([(x_{1}, 0)]) = [(x_{1}, 0)]$. Sinon on a:
  \begin{align*}
    g([(x_{1}^{1/3}, x_{2}^{1/3})]) &= [((x_{1}^{1/3} / x_{2}^{1/3})^{3}, 1)]
                                      \quad \text{$g$ définition} \\
                                    &= [(x_{1} / x_{2}, 1)] \\
                                    &= [(x_{1}, x_{2})]
                                     \quad \text{classe d'équivalence définition}.
  \end{align*}

  $g$ est automatiquement une collinéation car trois vecteurs quelconques sont
  toujours linéairement dépendants dans $\R^2$.

  Montrons maintenant que $g$ n'est pas induite par une application linéaire.
  Supposons par l'absurde que $g$ est induite par l'application linéaire
  $T : \R^{2} \to \R^{2}$. Alors on a $T(1, 0) = (k_{1}, 0)$ et
  $T(0, 1) = (0, k_{2})$ pour $k_{1}, k_{2}$ non-nuls. Soit $x_{1}$ un réél.
  D'un part, on a $T(x, 1) = k_{x} (x^{3}, 1)$ pour un $k_{x}$ non-nul, et
  d'autre part, on a $T(x, 1) = x T(1, 0) + T(0, 1)$. Donc, on a:
  $$
  k_{x} (x^{3}, 1) = x (k_{1}, 0) + (0, k_{2})
  $$
  Par conséquent, on obtient l'égalité des polynômes $k_{2}x^{3} = k_{1}x$ qui
  entraîne $k_{1} = k_{2} = 0$. Ceci oblige $T = 0$ qui est absurde.
\end{exm}
% Pour arriver à définir les endomorphisms, on note les définitions de
% fonction partielle, de noyau et domaine d'un fonction partielle, de
% sous-espace d'un géométrie projective, de morphisme et d'hyperplan.
% \begin{defn}
%   Une \textit{fonction partielle} de $X$ dans $Y$ est une fonction
%   $f: X \setminus N \to Y$ définie sur le complément d'un
%   sous-ensemble $N \subseteq X$. Si $f$ est une fonction partielle de
%   $X$ dans $Y$, on écrit $f: X \partialto Y$, ou encore $f: X \to Y$
%   si on sait que $N = \emptyset$. L'ensemble $N$ est appelé
%   \textit{noyau} de $f$ et sera désigné par $\kernel f$. L'ensemble
%   $X \setminus N$ est appelé \textit{domaine} de $f$ et sera désigné
%   par $\domain f$.
% \end{defn}
% \begin{defn}
%   Un \textit{sous-espace} d'un géométrie projective $G$ est un
%   sous-ensemble $E \subseteq G$ satisfaisant :
%   \begin{equation*}
%     \label{subspace}
%     a,b \in E \implies a \star b \subseteq E
%   \end{equation*}
% \end{defn}
% \begin{defn}
%   Un \textit{morphisme} d'un géométrie projective $G_1$ dans un
%   géométrie projective $G_2$ est une fonction partielle $g: G_1
%   \partialto G_2$ satisfaisant les axiomes suivant :
%   \begin{itemize}[align=left]
%     \item[($\text{M}_1$)] $\kernel g$ est un sous-espace de $G_1$,
%     \item[($\text{M}_2$)] si $a, b \not\in N$, $c \in N$ et $a \in b
%     \star c$, alors $ga = gb$,
%     \item[($\text{M}_3$)] si $a, b, c \not\in N$ et $a \in b \star c$,
%     alors $ga \in gb \star gc$.
%   \end{itemize}
% \end{defn}
% \begin{defn}
%   Un \textit{hyperplan} d'une géométrie projective $G$ est un
%   sous-espace $H$ de $G$ qui est maximal parmi les sous-espaces
%   stricts de $G$.
% \end{defn}
% Nous définissons enfin les endomorphismes, ainsi que le centre et
% l'axe d'un endomorphisme.
% \begin{defn}
%   Un \textit{endomorphisme} d'un géométrie projective $G$ est un
%   morphisme $\varphi: G \partialto G$. Un \textit{centre} d'un
%   endomorphisme $\varphi: G \partialto G$ est un point $z \in G$ tel
%   que $\varphi x \in x \star z$ pour tout $x \in \domain \varphi$. Un
%   \textit{axe} d'un endomorphisme $\varphi: G \partialto G$ est un
%   hyperplan $H \subseteq G$ tel que $\varphi x = x$ pour tout $x \in H
%   \cap \domain \varphi$.
% \end{defn}
\chapter{Formalisation}
Commençons par transcrire la définition de la géométrie projective
(\cref{projective-geometry}). Il faut d'abord représenter d'une certaine manière
les points, c'est-à-dire les éléments d'une géométrie projective, et la relation
de colinéarité. Les points de la géométrie projective seront abstraits,
c'est-à-dire un \textit{type}. La relation de colinéarité peut être représentée
par une fonction qui prend trois points et renvoie vrai ou faux ; après tout,
lorsque l'on dit qu'une relation $\ell$ est valable, ce que l'on fait, c'est
prendre trois points et se demander s'ils sont colinéaires ou non.
\begin{leancode}
class HasCollinear (P : Type) where
ell : P → P → P → Prop

export HasCollinear (ell)

variable {Point : Type} [HasCollinear Point]
\end{leancode}
Ici, la flèche entre les types \texttt{P} peut être considérée comme
la flèche entre le domaine et le codomaine dans les définitions de
fonctions dans le langage habituel des mathématiques, c'est-à-dire le
symbole \say{$\to$}.

Comme le point reste abstrait, il a fallu dire à Lean que l'existence
d'une interprétation de la colinéarité est assurée, c'est-à-dire que
la colinéarité des points est quelque chose que l'on peut décider.
Dans le code, cela est satisfait par la dernière ligne :
\texttt{Point} est un type dont la colinéarité existe.

Maintenant, on peut énoncer les axiomes $\text{L}_1$, $\text{L}_2$ et
$\text{L}_3$ dans la \cref{projective-geometry}.
\begin{leancode}
axiom l1 (a b : Point) : ell a b a
axiom l2 (a b p q : Point) : ell a p q → ell b p q → p ≠ q → ell a b p
axiom l3 (a b c d p: Point) : ell p a b → ell p c d →
∃ q : Point , ell q a c ∧ ell q b d
\end{leancode}
Il y a quelques points à noter ici. Tout d'abord, les flèches
utilisées ici sont différentes de celles que nous avons utilisées
ci-dessus, même si elles sont représentées par le même caractère
typographique : Il s'agit de flèches d'implication, pour lesquelles
nous utilisons normalement le symbole
\say{$\Longrightarrow$}. Deuxièmement, les conditions des axiomes qui
contiennent des conjonctions logiques sont converties en implications
sérielles. Expliquons cette équivalence logique. Supposons qu'on a une
condition de la forme $(p \land q) \implies r$. On a des équivalences
suivantes:
\begin{align*}
  &\iff \neg(p \land q) \lor r \\
  &\iff (\neg p \lor \neg q) \lor r \\
  &\iff \neg p \lor (\neg q \lor r) \\
  &\iff p \implies (\neg q \lor r) \\
  &\iff p \implies (q \implies r)
\end{align*}
Grâce à ces axiomes, nous pouvons prouver la proposition suivante.
\begin{prop}
  Toute relation ternaire $\ell$ qui satisfait les deux axiomes
  $\text{L}_1$ et $\text{L}_2$ est symétrique.
\end{prop}
Dans la preuve, on va dériver la colinéarité de toutes les
permutations possibles de trois points $a, b, c$ à partir de
$\ell(a, b, c)$. Ici, on ne donnera que $\ell(a, c, b)$ et les cinq
autres versions seront omises.
\begin{leancode}
theorem acb (a b c : Point) : ell a b c → ell a c b := by
intro acb_col -- Supposons que $\ell(a,b,c)$.
obtain rfl | bc_neq := eq_or_ne b c -- Si $b = c$,
exact acb_col -- le résultat est trivial,
apply l2 a c b c -- sinon on applique $\text{L}_2$;
exact acb_col -- première condition de $\text{L}_2$
apply l1 c b -- deuxième condition de $\text{L}_2$
exact bc_neq -- troisième condition de $\text{L}_2$.
\end{leancode}
Ce code est la transcription exacte de la phrase \say{Si $b = c$, le résultat
  est trivial, et sinon on applique $\text{L}_2$ à $\ell(a,b,c)$ et
  $\ell(c,b,c)$}.
\section{Regrouper le point et les axiomes}
Cette proposition la plus simple peut être considérée comme un premier théorème
prouvé avec cette configuration. Des théorèmes plus ambitieux peuvent également
être prouvés, mais décrire une géométrie projective en déclarant son point et
axiomes les uns après les autres n'est pas la manière la plus élégante. Pour
l'instant, ils n'ont qu'une relation sérielle dans le fichier. Ce qui est plus
utile, c'est une structure qui contiendrait toutes les informations nécessaires
à une géométrie projective. La structure suivante répond à ce besoin :
\begin{leancode}
class ProjectiveGeometry (point : Type u) where
ell : point → point → point → Prop
l1  : ∀ a b, ell a b a
l2  : ∀ a b p q, ell a p q → ell b p q → p ≠ q → ell a b p
l3  : ∀ a b c d p, ell p a b → ell p c d → ∃ q : point, ell q a c ∧ ell q b d
\end{leancode}
De cette manière, on a regroupé la relation de colinéarité et les axiomes
(\say{bundling}), et le point est resté dégroupé (\say{unbundled}) parce qu'il
s'agit d'un paramètre et que l'on veut que différents types de points
correspondent à différentes géométries projectives. Pour mieux expliquer le
contraste, on ne veut pas que des relations de colinéarité différentes
correspondent à des géométries projectives différentes ; on pense à une relation
de colinéarité canonique pour obtenir une géométrie projective. Ceci est
expliqué en détail
\href{https://leanprover.zulipchat.com/#narrow/stream/113489-new-members/topic/Creating.20a.20higher.20level.20object/near/423444220}{ici}.
\section{Instances}
Il est maintenant possible de déclarer que la projectivisation d'un espace
vectoriel est une géométrie projective.
\begin{leancode}
instance : ProjectiveGeometry (ℙ K V) :=
⟨
fun X Y Z => ¬ Independent ![X, Y, Z],
sorry,
sorry,
sorry
⟩
\end{leancode}
Expliquons comment cela fonctionne : Tout d'abord, une relation de colinéarité
sera fournie. Celle-ci est déjà établie mathématiquement dans la \cref{example}.
La fonction \texttt{Independent} provient de mathlib4. Elle prend trois points
de la projectivisation et affirme que leurs représentants sont linéairement
dépendants. Sur le plan technique, il prend une famille de points de
projectivisation. Une famille est une fonction d'indices dans des points.
Heureusement, Lean dispose d'une syntaxe pour convertir une liste en une telle
fonction : \say{\texttt{!}}. Si elle n'existait pas, on devrait l'écrire
manuellement :
\begin{leancode}
def indexer_generator (X Y Z : ℙ K V) : Fin 3 → ℙ K V :=
fun i : Fin 3 => match i with
| 0 => X
| 1 => Y
| 2 => Z
\end{leancode}

Ensuite, les axiomes seront vérifiés. Lean remplacera la relation de colinéarité
dans les axiomes pour nous. Les \say{sorry} servent de substitut à des preuves
qui n'ont pas encore été écrites.

Ainsi, nous pouvons continuer à étudier le côté purement synthétique et nos
résultats s'appliqueront automatiquement à la projectivisation d'un espace
vectoriel : Le côté concret. Pour les choses qui sont spécifiques aux
projectivisations, nous pouvons bien sûr passer du côté concret et continuer à
développer des propriétés pour elles.
% \section{Prochaines Étapes}
% Tout d'abord, la preuve en langage humain du théorème sera
% minutieusement assimilée. Pour cela, le livre "Modern Projective
% Geometry" \cite{ff00} de Claude-Alain Faure et Alfred Frölicher sera
% utilisé. Ce livre offre un cadre complet pour l'étude des géométries
% projectives. En particulier, le théorème de
% Desargues est compris en
% termes d'endomorphismes des géométries projectives et la preuve est
% effectuée en conséquence.

% Deuxièmement, l'existence des transcriptions de la machinerie requise,
% dans mathlib4 sera vérifiée. Quelques exemples seraient la définition
% de la projectivisation, des morphismes entre des espaces projectifs,
% etc.

% Enfin, le schéma complet de la preuve sera transcrit. Cela nécessitera
% probablement une bonne maîtrise de la programmation en Lean, et il
% est prévu que cette exigence soit satisfaite en cours de route.
\clearpage
\nocite{*}
\printbibliography[title=Références,heading=subbibliography]
\end{document}
