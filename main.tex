\documentclass[12pt, oneside]{memoir}

\newlength{\pagew}
\newlength{\pageh}
\setlength{\pagew}{210mm}
\setlength{\pageh}{297mm}

\setstocksize{\pageh}{\pagew}
\settrimmedsize{\pageh}{\pagew}{*}
\settypeblocksize{250mm}{150mm}{*}
\setlrmargins{*}{*}{1}
\setulmargins{*}{*}{1}
\setmarginnotes{0cm}{0cm}{0cm}
\checkandfixthelayout

%\usepackage[pass, showframe]{geometry}

\usepackage{mathpazo}
\usepackage{tgpagella}
\usepackage[T1]{fontenc}

\usepackage{mathtools}
\usepackage{amsmath}
\usepackage{amsthm}
\usepackage[mathscr]{eucal}

\usepackage{graphicx}
\usepackage{svg}

\usepackage{enumitem}
\usepackage{tabto}
\usepackage{lipsum}

\usepackage{biblatex}
\addbibresource{main.bib}

\usepackage{tikz}

\newtheorem{thm}{Théorème}
\newtheorem{prop}{Proposition}
\theoremstyle{definition}
\newtheorem{defn}{Définition}

% Unbutton the vector space from its origin.
\newcommand{\unbutton}[1][.6]{\mathbin{\vcenter{\hbox{\scalebox{#1}{$\bullet$}}}}}

% Definitional equal.
\newcommand{\defeq}{\vcentcolon=}

% Explicit set and set comprehension.
\newcommand{\set}[1]{\{ #1 \}}
\newcommand{\setcomp}[2]{\set{ #1 \,|\, #2 }}

% Projectivization.
\newcommand{\proj}{\boldsymbol{\mathscr{P}}}

% Disjoint union.
\newcommand{\discup}{\mathop{\dot{\cup}}}

% Partial map.
\newcommand*{\DashedArrow}[1][]{\mathbin{\tikz [baseline=-0.25ex,-latex, dashed,#1] \draw [#1] (0pt,0.5ex) -- (1.3em,0.5ex);}}%
\newcommand{\partialto}{\DashedArrow[->,dash pattern=on 4pt off 2pt]}

% Kernel and domain.
\DeclareMathOperator{\kernel}{Ker}
\DeclareMathOperator{\domain}{Dom}

\begin{document}
\thispagestyle{empty}
\begin{center}
  \vspace*{\fill}
  % {%
  % \setlength{\fboxsep}{0pt}%
  % \fbox{\includesvg[width=3cm]{logo.svg}}%
  % }
  \includesvg[width=2.5cm]{logo.svg}

  \vspace{1cm}
  \large{\textbf{PROJET DE FIN D'ÉTUDES}}

  \vspace{0.5cm}
  {\small pour obtenir le diplôme de}

  \vspace{0.5cm}
  l'\textsc{\textbf{Université Galatasaray}}

  {\small Spécialité : \textbf{Mathématiques}}

  \vspace{2.25cm}
  {\Large\textbf{Formalisation du théorème de Desargue en Lean 4}}

  Rapport 1

  \vspace{1.25cm}
  Préparé par \textbf{Abdullah Uyu}

  Résponsable : \textbf{Can Ozan Oğuz}

  \vspace{2.25cm}
  \textit{5 novembre 2023}
  \vspace*{\fill}
\end{center}
\clearpage
\pagenumbering{arabic}
\section*{Introduction}
Lean 4 est un assistant de preuve. La transcription d'une preuve en
langage humain dans un assistant de preuve est appelée
\textit{formalisation}. Lean 4 dispose d'une grande bibliothèque de
preuves, y compris de nombreuses preuves de licence, appelée
mathlib4. Dans cette bibliothèque, il y a d'importants théorèmes
manquants, et le théorème de Desargue est l'un d'entre eux.
\section*{Objectifs du Projet}
Ce projet vise à formaliser le théorème de Desargue dans Lean 4, en
trois étapes. Tout d'abord, la preuve en langage humain du théorème
sera minutieusement assimilée. Pour cela, le livre "Modern Projective
Geometry" de Claude-Alain Faure et Alfred Frölicher sera utilisé. Ce
livre offre un cadre complet pour l'étude des géométries
projectives. En particulier, le théorème de Desargue est compris en
termes d'endomorphismes des géométries projectives et la preuve est
effectuée en conséquence. Deuxièmement, l'existence des transcriptions
de la machinerie requise, dans mathlib4 sera vérifiée. Quelques
exemples seraient la définition de la projectivisation, des fonctions
partielles, etc. Enfin, le schéma complet de la preuve sera
transcrit. Cela nécessitera probablement une bonne maîtrise de la
programmation en Lean 4, et il est prévu que cette exigence soit
satisfaite en cours de route.
\section*{Résultats Préliminaires}
Tout le matériel mentionné dans cette section est tiré du
livre susmentionné. Il est à noter qu'ils étaient à l'origine en
anglais et qu'ils ont donc été traduits.  Notons tout d'abord la
définition d'une géométrie projective.
\begin{defn}
  Une \textit{géométrie projective} est un ensemble $G$ accompagné
  d'une relation ternaire $\ell \subseteq G \times G \times G$ telle
  que les axiomes suivants sont satisfaits :
  \begin{itemize}[align=left]
  \item[($\text{L}_1$)] $\ell(a,b,a)$ pour tout $a, b \in G$.
  \item[($\text{L}_2$)] $\ell(a,p,q)$, $\ell(b,p,q)$ et $p \neq q$
    $\implies$ $\ell(a,b,p)$.
  \item[($\text{L}_3$)] $\ell(p,a,b)$ et $\ell(p,c,d)$ $\implies$
    $\ell(q,a,c)$ et $(\ell(q,b,d)$ pour un $q \in G$.
  \end{itemize}
  Les éléments de $G$ sont appelés les \textit{points} de la
  géométrie. Et trois points $a, b, c$ sont dits \textit{colinéaires}
  si $\ell(a,b,c)$.
\end{defn}
Un système équivalent d'axiomes utilisant l'ensemble, par exemple,
$a \star b$, créé par $\ell$ est omis ici, par souci de concision. La
définition de morphisme sera donnée à l'aide de ce système.

Notons l'exemple le plus important pour une géométrie projective. En
fait, la formalisation sera effectuée avec cet exemple mais pas avec
la définition abstraite ci-dessus. De plus, la version du théorème
dans la géométrie euclidienne ne sera qu'une instance du théorème sur
cet exemple.
\begin{prop}
  Soit $V$ un espace vectoriel. Sur
  $V^{\unbutton} \defeq V \setminus \set{0}$, on définit une relation
  binaire comme suit: $x \sim y$ ssi $x, y$ sont linéairement
  indépendants. Comme ceci est une relation d'équivalence, l'ensemble
  quotient $\proj(V) \defeq V^{\unbutton} / \sim$ est bien défini et
  devient une géométrie projective si pour tout élément
  $X, Y, Z \in \proj(V)$ on définit : $\ell(X,Y,Z)$ ssi $X, Y, Z$ ont
  des représentants $x, y, z$ linéairement dépendants.
\end{prop}
Notons maintenant la définition d'un isomorphisme de géométries
projectives et une proposition d'isomorphisme.
\begin{defn}
  Un \textit{isomorphisme} de géométries projectives est une bijection
  $g: G_1 \to G_2$ satisfaisant $\ell_1(a,b,c)$ ssi
  $\ell_2(ga,gb,gc)$. Si $G_1 = G_2$, alors on dit que $g$ est une
  \textit{collinéation}.
\end{defn}
\begin{prop}
  Pour un $K$-espace vectoriel $V$, il existe une bijection naturelle
  \begin{equation*}
    \label{natural-bijection}
    s: \proj(V \times K) \to V \discup \proj(V)
  \end{equation*}
  induite par la fonction
  $t: (V \times K)^{\unbutton} \to V \discup \proj(V)$ définie par
  $t(x,\xi) = \xi^{-1}x$ if $\xi \neq 0$ et $t(x,0) = [x]$ pour
  $x \neq 0$, où $[x]$ désigne le point de $\proj(V)$ représenté par
  $x$. Il existe donc une relation ternaire unique $\overline{\ell}$ sur
  $V \discup \proj(V)$ pour laquelle $V \discup \proj(V)$ devient une
  géométrie projective et $s$ un isomorphisme. De plus, on a:
  \NumTabs{2}
  \begin{enumerate}[align=left]
  \item $\overline{\ell}(x, y, z)$ \tabto{3cm} ssi $x-z$ et $y-z$ sont linéairement
    dépendants dans $V$,
  \item $\overline{\ell}(x, y, [z])$ \tabto{3cm} ssi $x-y = \mu z$ pour un $\mu \in
    K$
  \item $\overline{\ell}(x, [y], [z])$ \tabto{3cm} ssi $[y] = [z]$,
  \item $\overline{\ell}([x], [y], [z])$ \tabto{3cm} ssi $\ell([x], [y], [z])$.
  \end{enumerate}
\end{prop}
Pour arriver à définir les endomorphisms, on note les définitions de
fonction partielle, de noyau et domaine d'un fonction partielle, de
sous-espace d'un géométrie projective, de morphisme et d'hyperplan.
\begin{defn}
  Une \textit{fonction partielle} de $X$ dans $Y$ est une fonction
  $f: X \setminus N \to Y$ définie sur le complément d'un
  sous-ensemble $N \subseteq X$. Si $f$ est une fonction partielle de
  $X$ dans $Y$, on écrit $f: X \partialto Y$, ou encore $f: X \to Y$
  si on sait que $N = \emptyset$. L'ensemble $N$ est appelé
  \textit{noyau} de $f$ et sera désigné par $\kernel f$. L'ensemble
  $X \setminus N$ est appelé \textit{domaine} de $f$ et sera désigné
  par $\domain f$.
\end{defn}
\begin{defn}
  Un \textit{sous-espace} d'un géométrie projective $G$ est un sous-ensemble $E
  \subseteq G$ satisfaisant:
  \begin{equation*}
    \label{subspace}
    a,b \in E \implies a \star b \subseteq E
  \end{equation*}
\end{defn}
\begin{defn}
  Un \textit{morphisme} d'un géométrie projective $G_1$ dans un
  géométrie projective $G_2$ est une fonction partielle $g: G_1
  \partialto G_2$ satisfaisant les axiomes suivant:
  \begin{itemize}[align=left]
  \item[($\text{M}_1$)] $\kernel g$ est un sous-espace de $G_1$,
  \item[($\text{M}_2$)] si $a, b \not\in N$, $c \in N$ et $a \in b
    \star c$, alors $ga = gb$,
  \item[($\text{M}_3$)] si $a, b, c \not\in N$ et $a \in b \star c$,
    alors $ga \in gb \star gc$.
  \end{itemize}
\end{defn}
\begin{defn}
  Un \textit{hyperplan} d'une géométrie projective $G$ est un
  sous-espace $H$ de $G$ qui est maximal parmi les sous-espaces
  stricts de $G$.
\end{defn}
Nous définissons enfin les endomorphismes, ainsi que le centre et
l'axe d'un endomorphisme.
\begin{defn}
  Un \textit{endomorphisme} d'un géométrie projective $G$ est un
  morphisme $\varphi: G \partialto G$. Un \textit{centre} d'un
  endomorphisme $\varphi: G \partialto G$ est un point $z \in G$ tel
  que $\varphi x \in x \star z$ pour tout $x \in \domain \varphi$. Un
  \textit{axe} d'un endomorphisme $\varphi: G \partialto G$ est un
  hyperplan $H \subseteq G$ tel que $\varphi x = x$ pour tout $x \in H
  \cap \domain \varphi$.
\end{defn}
\section*{Prochaines Étapes}
La dernière session d'étude a porté sur la compréhension de base des
morphismes, c'est pourquoi un approfondissement des morphismes est
prévu. Par exemple, le livre donne \textit{l'implosion} et
\textit{l'explosion} comme exemples d'endomorphismes. Comme la
compréhension visuelle est essentielle dans cette étude, la
visualisation de ces exemples apparaît comme une suite immédiate.
\nocite{*}
\printbibliography[title=Références,heading=subbibliography]
\end{document}
