\documentclass[12pt, oneside]{memoir}

\newlength{\pagew}
\newlength{\pageh}
\setlength{\pagew}{210mm}
\setlength{\pageh}{297mm}

\setstocksize{\pageh}{\pagew}
\settrimmedsize{\pageh}{\pagew}{*}
\settypeblocksize{250mm}{150mm}{*}
\setlrmargins{*}{*}{1}
\setulmargins{*}{*}{1}
\setmarginnotes{0cm}{0cm}{0cm}
\checkandfixthelayout

\usepackage{tgpagella}
\usepackage[T1]{fontenc}

\usepackage{amsthm}

\usepackage{svg}

\usepackage{lipsum}

\newtheorem{theorem}{Théorème}
\theoremstyle{definition}
\newtheorem{definition}{Définition}

\begin{document}
\thispagestyle{empty}
\vspace*{\fill}
\begin{center}
  \includesvg[width=3cm]{logo.svg} \\
  \vspace{1cm}
  \large{\textbf{PROJET DE FIN D'ÉTUDES}} \\
  \vspace{0.5cm}
  {\small pour obtenir le diplôme de} \\
  \vspace{0.5cm}
  l'\textsc{\textbf{Université Galatasaray}} \\
  {\small Spécialité : \textbf{Mathématiques}} \\
  \vspace{2.25cm}
  {\Large\textbf{Formalisation du théorème de Desargue en Lean 4}}\\
  Rapport 1 \\
  \vspace{1.25cm}
  Préparé par \textbf{Abdullah Uyu} \\
  Résponsable : \textbf{Can Ozan Oğuz} \\
  \vspace{2.25cm}
  \textit{5 novembre 2023}
\end{center}
\vspace*{\fill}
\clearpage
\pagenumbering{arabic}
\section*{Introduction}
\lipsum[1]
\section*{Objectifs du Projet}
\lipsum[2]
\section*{Résultats Préliminaires}
\lipsum[3]
\begin{definition}
\end{definition}
\begin{theorem}
\end{theorem}
\section*{Prochaines Étapes}
\lipsum[11]
\end{document}
